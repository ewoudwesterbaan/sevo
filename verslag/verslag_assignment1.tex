\documentclass[a4paper]{article}

\usepackage{xcolor}
\usepackage{fancyheadings}
\usepackage{listings}


\newcommand{\todo}[1]{\textcolor{red}{[#1]}}
\lhead{Open Universiteit}
\chead{IM0202, Software evolution}
\rhead{Assignment 1}

\begin{document}
\pagestyle{fancy}

\section*{Studentgegevens}
\begin{description}
	\item [Cursuscode] IM0202
	\item [Naam] \todo{invullen}
	\item [Studentnummer] \todo{invullen}
	\item [Naam] \todo{invullen}
	\item [Studentnummer] \todo{invullen}
\end{description}

\section*{Aanpak}
\todo{<Geef aan hoe jullie de opdracht hebben aangepakt en wie wat heeft gedaan, maximaal 1 A-4. Geef expliciet aandacht aan de volgorde van activiteiten>}



\section{Keuzen}
\subsection{Lines of code}
Dit is van belang voor de metrieken volume en unitsize. Beiden maken gebruik van regels code die geen commentaar zijn. Omdat dit een gemeenschappelijke logica is, is deze gedefinieerd in een Utils module.
Regels code worden nu beschouwd als alle niet lege regels minus de commentaar regels.

We beschouwen een regel commentaar als deze begint met \lstinline{//}. Voor blokcommentaar is de eerste gedacht om zo ook rgeles die starten met \lstinline{/*}, \lstinline{*} of \lstinline{*/} te beschouwen als commentaar. Dit gaat alleen niet helemaal op. Listing \ref{lst:asterixcmt} is valide java code en regel 2 begint met een \lstinline{*}.

\begin{lstlisting}[caption={* als commentaar beschouwen},label={lst:asterixcmt},escapechar=|, frame = single]
int i = 1
* 2;
\end{lstlisting}

We zouden kunnen zeggen dat de volume het aantal LOC van de methodes is. Maar dan missen we de velden en de Instance Initialization Blocks (wat geen constructors zijn!).


\section{Sourcecode}

\end{document}
